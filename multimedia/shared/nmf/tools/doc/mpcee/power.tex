\chapter{Power}
\section{Idle mode}
This is the default low power mode mode. It will simply consist to stop MMDSP
core clocks and use MMDSP emulation unit. There is no need of user intervention
to use this mode and wakeup latency is very small.

We enter this mode when no more reaction are available for running. This simply
consist of a call to SWITCH\_IDLE\_MODE() macro (Listing \ref{switchIdleMode}).

\begin{lstlisting}[caption=SWITCH\_IDLE\_MODE() macro
definition,label=switchIdleMode]
#define SWITCH_IDLE_MODE()                          \
{                                                   \
    wnop();wnop();                                  \
    EMU_unit_stop = 1;                              \
    wnop();wnop(); wnop(); wnop();                  \
}
\end{lstlisting}

Before calling SWITCH\_IDLE\_MODE() interrupt are mask. When we execute
\verb#EMU_unit_stop = 1;# MMDSP clocks are stopped, interrupts are unmask. Then
any interrupt will restart MMDSP clocks and code will continue to run after
SWITCH\_IDLE\_MODE() call.

\section{Ulp mode}
Ultra Low Power was originally develop for low power mp3 playback use case. It
consist to power off MMDSP core and put MMDSP TCM in retention mode. Wakeup only
occur when an HSEM interrupt or Fifo empty interrupt is received. Latency is
also greater than Idle mode since PRCMU need to restart power and clock of MMDSP
and MMDSP has to reboot before entering scheduler loop. Call to ulp mode logic
is done before calling SWITCH\_IDLE\_MODE() macro. In case Ulp mode is choosen
the flow control is broken and we will wakeup with a MMDSP reboot. User can
allow or prevent ULP mode.

\subsection{Ulp entry}
To decide if we enter in ulp mode MPCEE maitain sleepCounter global that can be
indrement with preventSleep() and decrement with allowSleep(). Initial
sleepCounter value after a reboot is 0.
forceWakeup global is also use to allow or not ulp mode and is use by CM driver
to be sure MMDSP stay awake to allow memory manipulation. This variable is
modify by ulp force wakeup and allow sleep services.

If we enter ulp logic with a negative sleepCounter value and a forceWakeup value
of 0 then we enter in ulp entry squence:
\begin{itemize}
  \item deactivate irq.
  \item call MPCEE and user register save callback. This allow driver to save
  hardware configuration.
  \item save MMDSP hardware configuration (YMEM, cache setup).
  \item set isSleeping global to 1.
  \item notify PRCMU to put MMDSP in retention mode.
  \item wait in a forever loop for the MMDSP to be shutdown.
\end{itemize}

\subsection{Ulp exit}
After PRCMU has power on MMDSP we start as a cold boot, it start to execute
instruction located at address 0. Startup code will detect it's a ulp mode exit
using isSleeping variable and jump into sleep\_wakeup() method which will
perform following sequence:
\begin{itemize}
  \item setup MMDSP core mode.
  \item init system stack and system stack checker.
  \item restore MMDSP hardware configuration (YMEM, cache setup).
  \item set isSleeping global to 0.
  \item set sleepCounter global to 0.
  \item setup interrupt registers.
  \item call MPCEE and user register restore callback.
  \item return to scheduler
\end{itemize}
