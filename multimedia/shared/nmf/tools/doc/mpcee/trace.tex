\chapter{Trace}
\section{Stm trace}
Stm hardware ip will be use to output various information. Channel number 100
will be use to output MPCEE traces.
\subsection{Activity trace}
Generated code will output components activity trace using:
\begin{itemize}
  \item nmfTraceActivity() : Will output trace of post into scheduler, start and
  end of reaction scheduling.
  \item nmfTraceCommunication() : Will output trace of reveive command from
  others core and send command to others core.
  \item nmfTracePanic() : Will output trace when a panic occur.
\end{itemize}
These trace can be deactivated to reduce MMDSP load using traceActive attribute.
CM driver will be responsible for that according to user request.

\subsection{User trace}
User can also embedded its own trace using the traceUser() api. That can help
debug in a non intrusive way. traceUser() take a single t\_uint32 parameter.

User trace can also be deactivate throught traceActive attribute.

\subsection{Profiling trace}
MPCEE implement \_\_mmdsp\_profile\_func\_enter() and
\_\_mmdsp\_profile\_func\_exit(). That allow instrumented code by the MMDSP
compiler to be output on stm bor debugging or performance analysis.

\section{Printf}
MPCEE also provide user with NmfPrint0(), NmfPrint1() and NmfPrint2() for
debugging purpose.

User output will then be redirected using interrupt 1 to the
CM driver which will provide a ring buffer readable by user code. This allow
MMDSP printf message to appear in rich OS logs.

These methods are highly intrusive since they wait for a CM driver acknoledge
before going further.
