\chapter{Services}
Services module implement code that will be call by CM driver to insure execution
of components lifecycle methods but also some MPCEE maintenance operation.
\section{Construct}
Construct services is always call after code and data component have been loaded
even if component does't implement lifecycle.constructor interface.

Before calling component constructor we invalidate instruction cache and then
data cache to insure component code and data use will not come from cache. Then
generic callMethodAndAck() function is call to execute constructer and
acknowledge communication.

\section{Starter, stopper and destructor}
These services are only call when component implement corresponding lifecycle.*
interface. Then generic callMethodAndAck() function is call to execute interface
method and acknowledge communication.

\section{Update stack}
Update stack service will be call each time CM driver detect we have to resize
component stack size (either to increase or to decrease it). In that case we
just call sched\_updateStack() with the new stack size as argument which will be
responsible to update stack checker hardware to detect stack overflow or
underflow.

\section{Lock and Unlock cache}
Lock and Unlock cache will be call each time CM driver as detected it need to
lock or unlock a MMDSP way after it has load/unload a new NMF component that
use code cache lock capability.

\section{Ulp force wakeup and allow sleep}
Ulp services will be call when CM driver is configure to use Ultra Low Power
features. This means that in that case MMDSP can be power off with TCM in
retention without CM driver being aware of this. So to be sure MMDSP is powered
when CM driver want to access MMDSP memory it use a force wakeup service, then
perform memory access and then allow MMDSP to sleep using allow sleep service.
